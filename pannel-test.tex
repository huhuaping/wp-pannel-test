% Options for packages loaded elsewhere
\PassOptionsToPackage{unicode}{hyperref}
\PassOptionsToPackage{hyphens}{url}
\PassOptionsToPackage{dvipsnames,svgnames,x11names}{xcolor}
%
\documentclass[
  12pt]{article}

\usepackage{amsmath,amssymb}
\usepackage{lmodern}
\usepackage{iftex}
\ifPDFTeX
  \usepackage[T1]{fontenc}
  \usepackage[utf8]{inputenc}
  \usepackage{textcomp} % provide euro and other symbols
\else % if luatex or xetex
  \usepackage{unicode-math}
  \defaultfontfeatures{Scale=MatchLowercase}
  \defaultfontfeatures[\rmfamily]{Ligatures=TeX,Scale=1}
\fi
% Use upquote if available, for straight quotes in verbatim environments
\IfFileExists{upquote.sty}{\usepackage{upquote}}{}
\IfFileExists{microtype.sty}{% use microtype if available
  \usepackage[]{microtype}
  \UseMicrotypeSet[protrusion]{basicmath} % disable protrusion for tt fonts
}{}
\makeatletter
\@ifundefined{KOMAClassName}{% if non-KOMA class
  \IfFileExists{parskip.sty}{%
    \usepackage{parskip}
  }{% else
    \setlength{\parindent}{0pt}
    \setlength{\parskip}{6pt plus 2pt minus 1pt}}
}{% if KOMA class
  \KOMAoptions{parskip=half}}
\makeatother
\usepackage{xcolor}
\setlength{\emergencystretch}{3em} % prevent overfull lines
\setcounter{secnumdepth}{5}
% Make \paragraph and \subparagraph free-standing
\ifx\paragraph\undefined\else
  \let\oldparagraph\paragraph
  \renewcommand{\paragraph}[1]{\oldparagraph{#1}\mbox{}}
\fi
\ifx\subparagraph\undefined\else
  \let\oldsubparagraph\subparagraph
  \renewcommand{\subparagraph}[1]{\oldsubparagraph{#1}\mbox{}}
\fi


\providecommand{\tightlist}{%
  \setlength{\itemsep}{0pt}\setlength{\parskip}{0pt}}\usepackage{longtable,booktabs,array}
\usepackage{calc} % for calculating minipage widths
% Correct order of tables after \paragraph or \subparagraph
\usepackage{etoolbox}
\makeatletter
\patchcmd\longtable{\par}{\if@noskipsec\mbox{}\fi\par}{}{}
\makeatother
% Allow footnotes in longtable head/foot
\IfFileExists{footnotehyper.sty}{\usepackage{footnotehyper}}{\usepackage{footnote}}
\makesavenoteenv{longtable}
\usepackage{graphicx}
\makeatletter
\def\maxwidth{\ifdim\Gin@nat@width>\linewidth\linewidth\else\Gin@nat@width\fi}
\def\maxheight{\ifdim\Gin@nat@height>\textheight\textheight\else\Gin@nat@height\fi}
\makeatother
% Scale images if necessary, so that they will not overflow the page
% margins by default, and it is still possible to overwrite the defaults
% using explicit options in \includegraphics[width, height, ...]{}
\setkeys{Gin}{width=\maxwidth,height=\maxheight,keepaspectratio}
% Set default figure placement to htbp
\makeatletter
\def\fps@figure{htbp}
\makeatother

\addtolength{\oddsidemargin}{-.5in}%
\addtolength{\evensidemargin}{-1in}%
\addtolength{\textwidth}{1in}%
\addtolength{\textheight}{1.7in}%
\addtolength{\topmargin}{-1in}%
\makeatletter
\makeatother
\makeatletter
\makeatother
\makeatletter
\@ifpackageloaded{caption}{}{\usepackage{caption}}
\AtBeginDocument{%
\ifdefined\contentsname
  \renewcommand*\contentsname{Table of contents}
\else
  \newcommand\contentsname{Table of contents}
\fi
\ifdefined\listfigurename
  \renewcommand*\listfigurename{List of Figures}
\else
  \newcommand\listfigurename{List of Figures}
\fi
\ifdefined\listtablename
  \renewcommand*\listtablename{List of Tables}
\else
  \newcommand\listtablename{List of Tables}
\fi
\ifdefined\figurename
  \renewcommand*\figurename{Figure}
\else
  \newcommand\figurename{Figure}
\fi
\ifdefined\tablename
  \renewcommand*\tablename{Table}
\else
  \newcommand\tablename{Table}
\fi
}
\@ifpackageloaded{float}{}{\usepackage{float}}
\floatstyle{ruled}
\@ifundefined{c@chapter}{\newfloat{codelisting}{h}{lop}}{\newfloat{codelisting}{h}{lop}[chapter]}
\floatname{codelisting}{Listing}
\newcommand*\listoflistings{\listof{codelisting}{List of Listings}}
\makeatother
\makeatletter
\@ifpackageloaded{caption}{}{\usepackage{caption}}
\@ifpackageloaded{subcaption}{}{\usepackage{subcaption}}
\makeatother
\makeatletter
\@ifpackageloaded{tcolorbox}{}{\usepackage[many]{tcolorbox}}
\makeatother
\makeatletter
\@ifundefined{shadecolor}{\definecolor{shadecolor}{rgb}{.97, .97, .97}}
\makeatother
\makeatletter
\makeatother
\ifLuaTeX
  \usepackage{selnolig}  % disable illegal ligatures
\fi
\usepackage[]{natbib}
\bibliographystyle{agsm}
\IfFileExists{bookmark.sty}{\usepackage{bookmark}}{\usepackage{hyperref}}
\IfFileExists{xurl.sty}{\usepackage{xurl}}{} % add URL line breaks if available
\urlstyle{same} % disable monospaced font for URLs
\hypersetup{
  pdftitle={Pannel test method},
  pdfkeywords={key, dictionary, word},
  colorlinks=true,
  linkcolor={blue},
  filecolor={Maroon},
  citecolor={Blue},
  urlcolor={Blue},
  pdfcreator={LaTeX via pandoc}}


\begin{document}


\def\spacingset#1{\renewcommand{\baselinestretch}%
{#1}\small\normalsize} \spacingset{1}


%%%%%%%%%%%%%%%%%%%%%%%%%%%%%%%%%%%%%%%%%%%%%%%%%%%%%%%%%%%%%%%%%%%%%%%%%%%%%%

\title{\bf Pannel test method}
\author{
Huhuaping 1\\
and\\Âuthóř 2\\
}
\maketitle

\bigskip
\bigskip
\begin{abstract}

\end{abstract}

\noindent%
{\it Keywords:} key, dictionary, word
\vfill

\newpage
\spacingset{1.9} % DON'T change the spacing!
\ifdefined\Shaded\renewenvironment{Shaded}{\begin{tcolorbox}[interior hidden, sharp corners, boxrule=0pt, frame hidden, borderline west={3pt}{0pt}{shadecolor}, breakable, enhanced]}{\end{tcolorbox}}\fi

\citep{pesaran2013} extends Pesaran's \textbf{CIP} test to the case of a
multifactor error structure. This is a non-trivial yet important
extension which is much more broadly applicable. It has also the
advantage of being intuitive and simple to implement. Following Bai and
Ng (2010) we also consider a panel unit root test based on simple
averages of cross-sectionally augmented Sargan--Bhargava type
statistics, which we denote by \textbf{CSB}. The presence of multiple
unobserved factors poses a number of additional challenges.

\hypertarget{panel-data-model-and-the-cips-test}{%
\section{Panel data model and the CIPs
test}\label{panel-data-model-and-the-cips-test}}

Let \(\Delta y_{it}\) be the observation on the \(i\)th cross section
unit at time \(t\), and suppose that it is generated as

\begin{equation}\protect\hypertarget{eq-dgp}{}{
\begin{aligned}
\Delta y_{i t}&=\beta_i\left(y_{i, t-1}-\boldsymbol{\alpha}_{i y}^{\prime} \mathbf{d}_{t-1}\right)+\boldsymbol{\alpha}_{i y}^{\prime} \Delta \mathbf{d}_t+u_{i t}, \\
i&=1,2, \ldots, N ; t=1,2, \ldots, T
\end{aligned}
}\label{eq-dgp}\end{equation}

Consider the following multifactor error structure
\begin{equation}\protect\hypertarget{eq-uit}{}{
\begin{aligned}
u_{i t}=\gamma_{i y}^{\prime} \mathbf{f}_t+\varepsilon_{i y t}
\end{aligned}
}\label{eq-uit}\end{equation}

This set up generalises Pesaran's \citeyearpar{pesaran2007} one factor
error specification (see Equation~\ref{eq-uit}).

The main assumptions on these error processes are:

\textbf{Assumption 1: Idiosyncratic Errors}. The idiosyncratic shocks
are independently distributed both across \(i\) and \(t\) , with zero
means, variances \(\sigma^2_i\), and finite fourth-order moments.

\textbf{Assumption 2: Factors}. The Factors vector follows a covariance
stationary process, with absolute summable autocovariances, distributed
independently of Idiosyncratic Errors for all \(i,t\) and \(t'\).

Combining Equation~\ref{eq-dgp} and Equation~\ref{eq-uit} it follows
that

\begin{equation}\protect\hypertarget{eq-yit-delta}{}{
\begin{aligned}
\Delta y_{i t}=\beta_i\left(y_{i, t-1}-\boldsymbol{\alpha}_{i y}^{\prime} \mathbf{d}_{t-1}\right)+\boldsymbol{\alpha}_{i y}^{\prime} \Delta \mathbf{d}_t+\boldsymbol{\gamma}_{i y}^{\prime} \mathbf{f}_t+\varepsilon_{i y t}
\end{aligned}
}\label{eq-yit-delta}\end{equation}

The hypothesis that all observed series,\(\Delta y_{it}\), have unit
roots and are not cross unit cointegrated can be expressed as

\begin{equation}\protect\hypertarget{eq-h0}{}{
H_0: \beta_i=0 \text { for all } i 
}\label{eq-h0}\end{equation}

against the alternative

\begin{equation}\protect\hypertarget{eq-h1}{}{
\begin{aligned}
H_1: \beta_i&<0 \text { for } i=1,2, \ldots, N_1 \\
\quad \beta_i&=0 \text { for } i=N_1+1, N_1+2, . ., N
\end{aligned}
}\label{eq-h1}\end{equation}

Under the null hypothesis, Equation~\ref{eq-yit-delta} can be solved for
to yield

\begin{equation}\protect\hypertarget{eq-yit}{}{
\begin{aligned}
&y_{i t}=y_{i 0}+\boldsymbol{\alpha}_{i y}^{\prime} \mathbf{d}_t+\boldsymbol{\gamma}_{i y}^{\prime} \mathbf{s}_{f t}+s_{i y t}, \\
&i=1,2, \ldots, N ; t=1,2, \ldots, T,
\end{aligned}
}\label{eq-yit}\end{equation}

where

\[
\mathbf{s}_{f t}=\mathbf{f}_1+\mathbf{f}_2+\cdots+\mathbf{f}_t \text {, and } 
s_{i y t}=\varepsilon_{i y 1}+\varepsilon_{i y 2}+\cdots+\varepsilon_{i y t}
\]

Suppose the \(k\times1\) vector of additional regressors follow the
general linear process

\begin{equation}\protect\hypertarget{eq-xit-delt}{}{
\begin{aligned}
\mathbf{x}_{i t}&=\mathbf{x}_{i 0}+\mathbf{A}_{i x} \mathbf{d}_t+\boldsymbol{\Gamma}_{i x} \mathbf{s}_{f t}+\mathbf{s}_{i x t}, \\
i&=1,2, \ldots, N ; t=1,2, \ldots, T,
\end{aligned}
}\label{eq-xit-delt}\end{equation}

Solving for \(\mathbf{x}_{i t}\)we have

\begin{equation}\protect\hypertarget{eq-xit}{}{
\begin{aligned}
\mathbf{x}_{i t} &=\mathbf{x}_{i 0}+\mathbf{A}_{i x} \mathbf{d}_t+\boldsymbol{\Gamma}_{i x} \mathbf{s}_{f t}+\mathbf{s}_{i x t}, \\
i &=1,2, \ldots, N ; t=1,2, \ldots, T,
\end{aligned}
}\label{eq-xit}\end{equation}

Combining Equation~\ref{eq-yit}, Equation~\ref{eq-xit} we obtain

\begin{equation}\protect\hypertarget{eq-zit}{}{
\mathbf{z}_{i t}=\mathbf{z}_{i 0}+\Gamma_i \mathbf{s}_{f t}+\mathbf{A}_i \mathbf{d}_t+\mathbf{s}_{i t}
}\label{eq-zit}\end{equation}

where
\(\mathbf{z}_{i t}=\left(y_{i t}, \mathbf{x}_{i t}^{\prime}\right)^{\prime}, \boldsymbol{\Gamma}_i=\left(\boldsymbol{\gamma}_{i y}, \boldsymbol{\Gamma}_{i x}^{\prime}\right)^{\prime}, \mathbf{A}_i=\left(\boldsymbol{\alpha}_{i y}, \mathbf{A}_{i x}^{\prime}\right)^{\prime}\),
and
\(\mathbf{s}_{i t}=\left(s_{i y t}, \mathbf{s}_{i x t}^{\prime}\right)^{\prime}\).

Averaging Equation~\ref{eq-zit} across \(i\) we obtain

\begin{equation}\protect\hypertarget{eq-zt-avr}{}{
\overline{\mathbf{z}}_t=\overline{\mathbf{z}}_0+\overline{\boldsymbol{\Gamma}}_{f t}+\overline{\mathbf{A}} \mathbf{d}_t+\overline{\mathbf{s}}_t
}\label{eq-zt-avr}\end{equation}

where
\(\overline{\mathbf{z}}_t=N^{-1} \sum_{i=1}^N \mathbf{z}_{i t}, \overline{\mathbf{A}}=N^{-1} \sum_{i=1}^N \mathbf{A}_i\),
and \(\overline{\mathbf{s}}_t=N^{-1}\sum_{i=1}^N \mathbf{s}_{i t}\).

Writing Equation~\ref{eq-yit-delta}, Equation~\ref{eq-zit} and
Equation~\ref{eq-zt-avr} in matrix notation, under the null for each
\(i\) we have

\begin{equation}\protect\hypertarget{eq-yi-mtrx}{}{
\begin{aligned}
&\Delta \mathbf{y}_i=\mathbf{F} \boldsymbol{\gamma}_{i y}+\Delta \mathbf{D} \boldsymbol{\alpha}_{i y}+\boldsymbol{\varepsilon}_{i y} 
\end{aligned}
}\label{eq-yi-mtrx}\end{equation}

\begin{equation}\protect\hypertarget{eq-zi-mtrx}{}{
\begin{aligned}
&\Delta \mathbf{Z}_i=\mathbf{F} \boldsymbol{\Gamma}_i^{\prime}+\Delta \mathbf{D} \mathbf{A}_i^{\prime}+\mathbf{E}_i 
\end{aligned}
}\label{eq-zi-mtrx}\end{equation}
\begin{equation}\protect\hypertarget{eq-zavr-mtrx}{}{
\begin{aligned}
&\Delta \overline{\mathbf{Z}}=\mathbf{F}^{\prime}+\Delta \mathbf{D} \overline{\mathbf{A}}^{\prime}+\overline{\mathbf{E}}
\end{aligned}
}\label{eq-zavr-mtrx}\end{equation}

In view of the above we shall base our test of the panel unit root
hypothesis on the -ratio of the Ordinary Least Squares (OLS) estimator
of in the following cross-sectionally augmented regression

\begin{equation}\protect\hypertarget{eq-reg-cs}{}{
\Delta y_{i t}=b_i y_{i t-1}+\mathbf{c}_i^{\prime} \overline{\mathbf{z}}_{t-1}+\mathbf{h}_i^{\prime} \Delta \overline{\mathbf{z}}_t+\mathbf{g}_i^{\prime} \mathbf{d}_t+\epsilon_{i t}
}\label{eq-reg-cs}\end{equation}

The \(t\)-ratio of \(\hat{b}_i\) is given by

\[
\begin{aligned}
t_i(N, T) &=\frac{\Delta \mathbf{y}_i^{\prime} \overline{\mathbf{M}} \mathbf{y}_{i,-1}}{\hat{\sigma}_i\left(\mathbf{y}_{i,-1}^{\prime} \overline{\mathbf{M}} \mathbf{y}_{i,-1}\right)^{1 / 2}} \\
&=\frac{\sqrt{T-2 k-5} \Delta \mathbf{y}_i^{\prime} \overline{\mathbf{M}} \mathbf{y}_{i,-1}}{\left(\Delta \mathbf{y}_i^{\prime} \overline{\mathbf{M}}_i \Delta \mathbf{y}_i\right)^{1 / 2}\left(\mathbf{y}_{i,-1}^{\prime} \overline{\mathbf{M}} \mathbf{y}_{i,-1}\right)^{1 / 2}}
\end{aligned}
\]

The panel unit root test can now be based on the average of the t-ratios

\begin{equation}\protect\hypertarget{eq-cips-nt}{}{
C I P S_{N T}=N^{-1} \sum_{i=1}^N t_i(N, T)
}\label{eq-cips-nt}\end{equation}

\hypertarget{cross-sectionally-augmented-sarganbhargava-testcsb}{%
\section{cross-sectionally augmented Sargan--Bhargava
Test(CSB)}\label{cross-sectionally-augmented-sarganbhargava-testcsb}}

The cross-sectional augmentation approach can also be exploited in the
case of other unit root tests, such as the test proposed by Sargan and
Bhargava (1983). In the single time series case, the Sargan-Bhargava
statistic was modified by Stock (1999) to allow for serial correlation.
This test has also been recently adopted by Bai and \(\mathrm{Ng}\)
(2010) in the panel context with good effects. Recall that the data
generating process for \(y_{i t}\) under the null is given by

\[
\Delta y_{i t}=\boldsymbol{\alpha}_{i y}^{\prime} \Delta \mathbf{d}_t+\boldsymbol{\gamma}_{i y}^{\prime} \mathbf{f}_t+\varepsilon_{i y t} .
\]

For each \(i\), the cross-sectionally augmented Sargan-Bhargava
statistic, is given by

\[
\operatorname{CSB}_i(N, T)=T^{-2} \sum_{t=1}^T \widehat{u}_{i t}^2 / \hat{\sigma}_i^2,
\]

where

\[
\widehat{u}_{i t}=\sum_{j=1}^t \hat{\varepsilon}_{i j}, \quad \text { and } \quad \hat{\sigma}_i^2=\sum_{t=1}^T \hat{\varepsilon}_{i t}^2 /[T-(k+1)]
\]

and \(\hat{\varepsilon}_{i t}\) are the OLS residuals from the
regressions of \(\Delta y_{i t}\) on \(\Delta \overline{\mathbf{z}}_t\),
in the case of models with an intercept only. If the underlying series
are trended, \(\hat{\varepsilon}_{i t}\) must be calculated from a
regression of \(\Delta y_{i t}\) on an intercept and
\(\Delta \overline{\mathbf{z}}_t\), with \(\hat{\sigma}_i^2\) computed
as \(\hat{\sigma}_i^2=\)
\(\sum_{t=1}^T \hat{\varepsilon}_{i t}^2 /[T-(k+2)]\). The use of
cross-sectional augmentation as a way of dealing with the unobserved
factors is justified using (17), which renders
\(\hat{\varepsilon}_{it}\) free of the nuisance parameters (namely the
factor loadings). It is now easy to prove that for each \(i\), the
\(\operatorname{CSB}_i(N, T)\) statistic converges to a functional of
Brownian motions, which is independent of the factors as well as their
loadings. The CSB test is then based on the cross-sectional average of
the \(\operatorname{CSB}_i(N, T)\) statistics, given by

\[
\operatorname{CSB}_{N T}=N^{-1} \sum_{i=1}^N \operatorname{CSB}_i(N, T)
\]

\hypertarget{the-case-of-residual-serial-correlation}{%
\section{The case of residual serial
correlation}\label{the-case-of-residual-serial-correlation}}

We relax Assumption 1, and consider the implications of residual serial
correlation for our proposed tests. In error factor models, residual
serial correlation can be modelled in a number of different ways,
directly via the idiosyncratic components, through the factor(s), or a
mixture of the two. We focus on the serial correlation in the
idiosyncratic errors and model the residual serial correlation as

\begin{equation}\protect\hypertarget{eq-res-zeta-iyt}{}{
\begin{aligned}
&\zeta_{iyt }=\theta_i \zeta_{iy, t-1}+\eta_{iyt}, \quad\left|\theta_i\right|<1, \\
&\text { for } i=1,2, \ldots, N ; \quad t=1,2, \ldots, T
\end{aligned}
}\label{eq-res-zeta-iyt}\end{equation}

where \(\zeta_{i y t}\) is the idiosyncratic component of
\(u_{i t}=\gamma_{i y}^{\prime} \mathbf{f}_t+\zeta_{i y t}\), and
\(\eta_{\text {iyt }}\) is independently distributed across both \(i\)
and \(t\), with zero means and variances,
\(0<\sigma_{i \eta}^2 \leq K\).

To keep the exposition simple we confine our analysis to the first order
stationary processes, though the analysis readily extends to higher
order processes. Under Equation~\ref{eq-res-zeta-iyt} we have

\begin{equation}\protect\hypertarget{eq-res-delta-yit}{}{
\Delta y_{i t}=\beta_i\left(y_{i, t-1}-\boldsymbol{\alpha}_{i y}^{\prime} \mathbf{d}_{t-1}\right)+\boldsymbol{\alpha}_{i y}^{\prime} \Delta \mathbf{d}_t+\gamma_{i y}^{\prime} \mathbf{f}_t+\zeta_{i y t}\left(\theta_i\right),
}\label{eq-res-delta-yit}\end{equation}

where
\(\zeta_{\text {iyt }}\left(\theta_i\right)=\left(1-\theta_i L\right)^{-1} \eta_{\text {iyt }}\).
We also assume the coefficients of the autoregressive process to be
homogeneous across \(i\), although this could be relaxed at the cost of
more complex mathematical details. Under the null that \(\beta_i=0\),
with \(\theta_i=\theta\) and \(\mathbf{d}_t=(1,0)^{\prime}\),
Equation~\ref{eq-res-delta-yit} reduces to

\begin{equation}\protect\hypertarget{eq-res-yit-reduce}{}{
\Delta y_{i t}=\gamma_{i y}^{\prime} \mathbf{f}_t+\zeta_{i y t}(\theta) \text {, }
}\label{eq-res-yit-reduce}\end{equation}

and upon using Equation~\ref{eq-res-zeta-iyt} under the null hypothesis
we have

\begin{equation}\protect\hypertarget{eq-res-yit-null}{}{
\Delta y_{i t}=\theta \Delta y_{i, t-1}+\gamma_{i y}^{\prime}\left(\mathbf{f}_t-\theta \mathbf{f}_{t-1}\right)+\eta_{i y t} .
}\label{eq-res-yit-null}\end{equation}

The individual CADF regressions can be written as

\begin{equation}\protect\hypertarget{eq-res-yit-ind}{}{
\Delta \mathbf{y}_i=b_i \mathbf{y}_{i,-1}+\overline{\mathbf{W}}_{i 1} \mathbf{h}_i+\epsilon_i, \quad \text { for } i=1,2, \ldots, N \text {, }
}\label{eq-res-yit-ind}\end{equation}

where
\(\overline{\mathbf{W}}_{i 1}=\left(\Delta \mathbf{y}_{i,-1}, \Delta \overline{\mathbf{Z}}, \Delta \overline{\mathbf{Z}}_{-1}, \boldsymbol{\tau}_T, \overline{\mathbf{Z}}_{-1}\right)\),
which is a \(T \times(3 k+5)\) matrix. The \(t\)-ratio of \(\hat{b}_i\)
in regression Equation~\ref{eq-res-yit-ind} is given by

\begin{equation}\protect\hypertarget{eq-res-ti-ratio}{}{
\begin{aligned}
t_i(N, T) &=\frac{\Delta \mathbf{y}_i^{\prime} \overline{\mathbf{M}}_{i 1} \mathbf{y}_{i,-1}}{\hat{\sigma}_i\left(\mathbf{y}_{i,-1}^{\prime} \overline{\mathbf{M}}_{i 1} \mathbf{y}_{i,-1}\right)^{1 / 2}} \\
&=\frac{\sqrt{T-(3 k+6)} \Delta \mathbf{y}_i^{\prime} \overline{\mathbf{M}}_{i 1} \mathbf{y}_{i,-1}}{\left(\Delta \mathbf{y}_i^{\prime} \overline{\mathbf{M}}_{i 1, p} \Delta \mathbf{y}_i\right)^{1 / 2}\left(\mathbf{y}_{i,-1}^{\prime} \overline{\mathbf{M}}_{i 1} \mathbf{y}_{i,-1}\right)^{1 / 2}}
\end{aligned}
}\label{eq-res-ti-ratio}\end{equation}

Empirical applications

\hypertarget{real-intrest-rates}{%
\section{real intrest rates}\label{real-intrest-rates}}

These results suggest that for a significant number of countries the
Fisher parity holds.This is inline with recent findings reported in
Westerlund(2008) using panel cointegration tests.

\hypertarget{real-equity-prices}{%
\section{real equity prices}\label{real-equity-prices}}

Overall, the test results are inline with the generally accepted view
that real equity prices approximately follow random walks with a drift.


  \bibliography{bib/method-unit-root-test.bib}


\end{document}
